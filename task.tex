\section{Постановка задачи}
\label{}

В качестве входных данных имеется географическая карта, закодированная по стандарту Geography Markup Language (GML). Этот стандарт предназначен для кодирования областей различной природы – к примеру, лесных массивов, водоемов или островов – в виде вложенных многоугольников при помощи XML-грамматики [ссылка]. 

Требуется преобразовать входные данные таким образом, чтобы одновременно выполнялись следующие условия:

\begin{enumerate}
\item Отсутствуют вложенные области, т.е. все области односвязны.
\item Число точек, задающих границу каждой области, не превышает заданной константы N.
\item Количество добавочных соединений и их длины минимальны.
\item Создание новых вершин запрещено.
\end{enumerate}

Возможность работы в реальном времени не требуется, поэтому оптимизация алгоритма в данной работе не затрагивается.

Понятно, что поддержка многосвязных карт автоматически означает и поддержку односвязных. Но обратное неверно. Для обратного преобразования нужно разрезать многосвязную карту, сделав ее односвязной.